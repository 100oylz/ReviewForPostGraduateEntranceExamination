\subsection{高斯消元法}

\index{齐次线性方程组}

\index{非齐次线性方程组}
\begin{definition}
    一般地我们把$AX=b$中如果b中所有元素至少有一个不为0,则称为\textcolor{red}{非齐次线性方程组},否则称为\textcolor{red}{齐次线性方程组},满足方程的一组X称为方程组的一组解
\index{高斯消元法}
\index{高斯消元法!矩阵的行列初等变化}
\textcolor{red}{矩阵的行列初等变化}如下:

\begin{enumerate}
    \item 交换两行(列)的位置
    \item 用一非零数数乘某一行(列)的所有元
    \item 把矩阵的某一行(列)的适当倍数加到另一行上
\end{enumerate}
\end{definition}


\begin{note}
    解线性方程组可以使用有限次的初等行变换操作增广矩阵
\end{note}

\begin{algorithm}[H]
    \caption[]{\textcolor{red}{高斯消元法}}
    $i\leftarrow 1$\\
    $j\leftarrow 1$\\
    首先将一般矩阵转化为行阶梯形矩阵
    \begin{enumerate}
        \item 对于第j列寻找非零元,如果存在,利用初等行变换,将他替换到第i行,如果不存在,j=j+1,继续执行1,直到j无法再增加
        \item 利用初等行变换,将i以下所有行第j列元素更新为0,第i行第j列元素更新为1,i++,j++
        \item 如果i无法再增加,结束循环,如果i可以再增加,转到1
    \end{enumerate}

    然后将行阶梯形矩阵转化为行简化阶梯型矩阵

    然后此时i和j,再进行自减,重复上面类似的操作,对于非零首元更新i以上的第j列元素为0
\end{algorithm}

如果一个矩阵每个非零元都出现再上一行非零首元的右边,则称为\textcolor{red}{行阶梯形矩阵}\index{高斯消元法!行阶梯形矩阵}

如果行阶梯形矩阵每个非零行的非零首元都是1,且非零首元所在列其余元都是0,则称为\textcolor{red}{行简化阶梯型矩阵}\index{高斯消元法!行简化阶梯型矩阵}

\begin{theorem}
    \index{线性方程组解的特征}

    \begin{center}
        \zihao{-4}{\textcolor{red}{线性方程组解的特征}}
    \end{center}

    对于一般的线性方程组,可以通过消元步骤将其化为行简化阶梯型矩阵,假设$\bar A=(A,b)$为方程组$AX=b$的行简化阶梯型矩阵:

\begin{equation*}
        \bar A=\begin{pmatrix}
            c_{11}&0&\cdots&0&c_{1,r+1}&\cdots&c_{1n}&d_1\\
            0&c_{21}&\cdots&0&c_{2,r+1}&\cdots&c_{2n}&d_2\\
            \vdots&\vdots&\ &\vdots&\vdots&\ &\vdots&\vdots\\
            0&0&\cdots&c_{rr}&c_{r,r+1}&\cdots&c_{r,n}&d_{r}\\
            0&0&\cdots&0&0&0&d_{r+1}\\
            \vdots&\vdots&\ &\vdots&\vdots&\ &\vdots&\vdots\\
            0&0&\cdots&0&0&\cdots&0&0\\
        \end{pmatrix}
\end{equation*}

如果$d_{r+1}\neq 0$,则方程组无解

如果$d_{r+1}=0$,则存在如下情况:

\begin{enumerate}
    \item r=n,存在唯一解
    $$
    \begin{aligned}
        \begin{cases}
            x_1\quad \quad \quad &=d_1,\\
            \quad x_2\quad \quad &=d_2,\\
            \quad \quad \cdots\cdots\quad &\\
            \quad\quad \quad  x_n&=d_n\\
        \end{cases}
    \end{aligned}
    $$

    \item 当r<n时,存在无穷多的解

    把矩阵的每一个非零首元所在列的元称为\index{线性方程组解的特征!基本未知量}\textcolor{red}{基本未知量},将其余元所在列称为\textcolor{red}{自由未知量}\index{线性方程组解的特征!自由未知量},基本变量由自由变量的线性组合表示。
\end{enumerate}
\end{theorem}

\index{齐次线性方程组!平凡解}

关于齐次线性方程组$AX=0$,总是存在零解(\textcolor{red}{平凡解}),X=O


当r<n时,存在无穷多的解,设m个n元方程组组成的线性方程组$AX=0$,若$m<n$,则方程组必有非零解

\index{矩阵等价}

若一个矩阵A能够通过有限次的初等变化变成B,则称A与B\textcolor{red}{等价},记作$A\cong B$,如果使用的时行(列)初等变化,则称为A与B行(列)等价

矩阵的等价关系性质如下:
\begin{theorem}
    \begin{itemize}
        \item {反身性:若$A\cong A$}
        \item {对称性:若$A\cong B$,则$B\cong A$}
        \item {传递性:若$A\cong B,B\cong C$,则$A\cong C$}
    \end{itemize}
\end{theorem}

\subsection{初等矩阵}
\begin{definition}
    \index{初等矩阵}

将单位矩阵作一个初等变换得到的矩阵,叫做\textcolor{red}{初等矩阵}
\end{definition}

对一个$m\times n$的矩阵而言,进行初等行变换相当于左乘一个初等矩阵,进行初等列变换相当于右乘一个初等矩阵

\begin{note}
    如果一个矩阵能够由A进行行初等变换得到,则必然存在有限个初等矩阵,使得
$$
B=E_kE_{k-1}\cdots E_1A
$$
如果一个矩阵能够由A进行列初等变换得到,则必然存在有限个初等矩阵,使得
$$
B=AE_1'E_2'\cdots E_k'
$$
如果一个矩阵能够由A进行初等变化得到,则必然存在有限个初等矩阵,使得
$$
B=P_kP_{k-1}\cdots P_1 A Q_1Q_2\cdots Q_l
$$
\end{note}
