\subsection{逆矩阵的概念与性质}
\begin{definition}{逆矩阵}{inversematrix}
	\index{逆矩阵}
	设A为n阶方阵,若存在n阶方阵B使得
	$$
		AB=BA=I
	$$

	则称A是可逆矩阵,简称A可逆,B为A的\textcolor{red}{逆矩阵}
\end{definition}

\begin{example}
	如果A可逆,A的逆矩阵唯一

	\begin{proof}
		假设存在B和C两个逆矩阵,根据性质可知:
		$$
			\begin{aligned}
				AB=        & BA=I        \\
				AC=        & CA=I        \\
				B=BI=B(AC) & =(BA)C=IC=C \\
			\end{aligned}
		$$
		由此可知B和C相等,A的逆矩阵唯一
	\end{proof}
\end{example}

由定义可知:如果A是B的逆矩阵,那么B也是A的逆矩阵,AB互为逆矩阵

\index{逆矩阵!对角矩阵的逆矩阵}

根据定义可以得知,若A为对角矩阵
$$diag(d_1,d_2,\cdots.d_n),d_i\neq 0$$
,则\textcolor{red}{对角矩阵的逆矩阵}为:

$$
	A^{-1}=diag(\frac{1}{d_1},\frac{1}{d_2},\cdots,\frac{1}{d_n})
$$

\textcolor{red}{并非每个矩阵都可逆}

\begin{theorem}{逆矩阵}{inversematrix}
	\index{逆矩阵!逆矩阵的性质}

	逆矩阵的性质,假设A与B均为n阶可逆矩阵,数$\lambda\neq 0$,则

	\begin{enumerate}
		\item {$A^{-1}$可逆,且$(A^{-1})^{-1}=A$}
		\item $\lambda A$可逆,且$(\lambda A)^{-1}=\frac{1}{\lambda}A^{-1}$
		\item (AB)可逆,且$(AB)^{-1}=B^{-1}A^{-1}$
		\item $A^T$可逆,且$(A^T)^{-1}=(A^{-1})^T$
	\end{enumerate}
\end{theorem}

下面分别证明3和4号定理:

\begin{example}
	假设A与B均为n阶可逆矩阵,(AB)可逆,且$(AB)^{-1}=B^{-1}A^{-1}$

	\begin{proof}
		$$
			AB(B^{-1}A^{-1})=A(BB^{-1})A^{-1}=AA^{-1}=I
		$$

		所以AB可逆,且$(AB)^{-1}=B^{-1}A^{-1}$
	\end{proof}
\end{example}

\begin{example}
	假设A为n阶可逆矩阵,$A^T$可逆,且$(A^T)^{-1}=(A^{-1})^T$

	\begin{proof}
		$$
			\begin{aligned}
				A^T(A^{-1})^T & =(A^{-1}A)^T\quad \because B^TA^T=(AB)^T \\
				              & =I^T                                     \\
				              & =I                                       \\
			\end{aligned}
		$$

		所以$A^T$可逆,且$(A^T)^{-1}=(A^{-1})^T$
	\end{proof}
\end{example}

其中3号定理可以推广到n个可逆矩阵的情况:

$$
	(A_1A_2\cdots A_n)^{-1}=A_n^{-1}A_{n-1}^{-1}\cdots A_1^{-1}  $$

\textcolor{red}{证明矩阵是否可逆的关键是证明一个矩阵是否存在另一个矩阵使得两个矩阵相乘为I}

以下是几道例题:

\begin{example}

	设方阵B为幂等矩阵($B^2=B$,从而$\forall  k\in N^*$\footnote[1]{$N^*$为正整数集合,即$N\backslash\{0\}$}),证明A可逆,且A的逆矩阵为:

	$$ A^{-1}=\frac{1}{2}(3I-A) $$

	\begin{proof}
		$$
			\begin{aligned}
				A(\frac{1}{2}(3I-A))              & =\frac{1}{2}(A-3A^2)          \\
				\because A=I+B \quad              & \therefore A^2=(I^2+2B+B^2)   \\
				\because B\mbox{是幂等矩阵}.\quad & \therefore A^2=(I+3B)=(3A-2I) \\
				A(\frac{1}{2}(3I-A))              & =\frac{1}{2}(3A-3A+2I)=I      \\
				\therefore A\mbox{可逆且}         & A^{-1}=\frac{1}{2}(3I-A)
			\end{aligned}
		$$
	\end{proof}
	\begin{note}
		若已知A的逆矩阵则可以直接相乘证明乘积为I
	\end{note}
\end{example}

\begin{example}
	设矩阵A满足$A^2-3A-10I=O$,证明A,A-4I都可逆

	\begin{proof}
		\par
		由$A^2-3A-10I=O$,可知$A(A-3I)=10O$,$A(\frac{1}{10}(A-3I))=I$,A可逆,且$A^-1=\frac{1}{10}(A-3I)$

		由$A^2-3A-10I=O$,可知$(A+I)(A-4I)=6I$,$(A-4I)(\frac{1}{6}(A+I))=I$,A-4I可逆,且$(A-4I)^-1=\frac{1}{6}(A+I)$

	\end{proof}

	\begin{note}
		若A的逆矩阵未知,则通过已知条件,凑出A乘另一个矩阵等于I的形式,则可以证明A可逆
	\end{note}
\end{example}

\begin{theorem}{可逆的等价命题}{inverseequal}
	设A为n阶矩阵,则下列命题是等价的:
	\begin{enumerate}
		\item  A是可逆的
		\item 齐次线性方程组AX=0只有零解
		\item A与I行等价
		\item A可表示为有限个初等矩阵的乘积
	\end{enumerate}
\end{theorem}

以下进行循环证明:
\begin{proof}
	\par
	$1\Rightarrow 2$

	$$
		\begin{aligned}
			 & AX=0                                           \\
			 & \because A\mbox{可逆},\therefore A^{-1}AX=IX=0 \\
			 & \therefore X=0
		\end{aligned}
	$$

	AX=0只有零解

	$2\Rightarrow 3$

	若齐次线性方程组AX=0只有零解

	\begin{figure}[H]
		\centering
		\begin{tikzpicture}[>=latex,scale=.75]
			\draw[->] (-2,0)node[left]{A}--node[above]{\textbf{行初等变化}}(2,0)node[right]{B(B为行阶梯形矩阵)};
		\end{tikzpicture}
	\end{figure}

	则AX=0与BX=0同解,假设B中的对角元\footnote{矩阵行列式中对角线上的元素}存在0,则根据定理\ref{thm:linearsymbol},存在无穷多的解,与原定义不符。所以B中对角元全不为0,A经过行初等变化后可以得到的行阶梯形矩阵为I,根据定义\ref{def:equivalence},则A与I行等价

	$3\Rightarrow 4$

A与I行等价,所以A经过初等行变换能够变成I,初等行变换相当于矩阵左乘有限多个初等矩阵,即存在有限多个E,使得
$$
\begin{aligned}
	(E_1E_2\cdots E_k)A=I\\
	A=I(E_K^{-1}\cdots E_2^{-1}E_1^{-1})\\
\end{aligned}
$$

\textcolor{red}{初等矩阵的逆仍为初等矩阵},所以A可表示为有限个初等矩阵的乘积

	$4\Rightarrow 1$

	A可表示为有限个初等矩阵的乘积,则对于初等矩阵$E_1,E_2,\cdots,E_k$:

	$$
	\begin{aligned}
		A=E_1E_2\cdots E_k I\\
		A(E_K^{-1}\cdots E_2^{-1}E_1^{-1})=I\\
	\end{aligned}
	$$

	A可逆

\end{proof}

\begin{lemma}
	如果非齐次线性方程组AX=b有唯一解的充分必要条件是A可逆

	\begin{proof}
		先证充分性,若A可逆,则AX=b有唯一解,且唯一解为$A^{-1}b$

		再证必要性,假设存在AX=b有唯一解,当时A不可逆,即对于AX=0存在一个非零解Z
		
		假设$X_1$为AX=b的唯一解,已知$X_2=X_1+Z$

		则$AX_2=A(X_1+Z)=b+O=b$0000000

		$X_2$也为AX=b的一个解,因此假设不成立,原命题成立
	\end{proof}
\end{lemma}

\subsection{用行初等变化求逆矩阵}

\begin{note}
	若对矩阵(A,I)施以行初等变换将A变为I,则I变为$A^{-1}$

	\begin{figure}[H]
		\centering
		\begin{tikzpicture}[>=latex,scale=1.25]
			\draw[->](-3,0)node[left]{(A,I)}--node[above]{行初等变化}(3,0)node[right]{$(I,A^{-1})$};
		\end{tikzpicture}
	\end{figure}
\end{note}